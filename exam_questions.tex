% !TeX spellcheck = russian-aot
% Commenting - Ctrl+T
\RequirePackage{cmap}
\documentclass[12pt, a4paper]{article}
\oddsidemargin=-5mm
\textwidth=17cm
\topmargin=0pt
\setlength{\textheight}{46\baselineskip}
\setlength{\textheight}{\baselinestretch\textheight}
\addtolength{\textheight}{\topskip}
%-------------
%\usepackage[UTF8]{inputenc}  
\usepackage[T2A]{fontenc} 
\usepackage[a4paper,hmargin=2.5cm,vmargin=2.5cm]{geometry}     
\usepackage[english,russian]{babel}
\frenchspacing   
\usepackage{latexsym} % \lhd \rhd
\usepackage{amsfonts} % \mathbb

\usepackage{xcolor}
\usepackage{hyperref}[]
\definecolor{linkcolor}{HTML}{000000} % ???? ??????
\definecolor{urlcolor}{HTML}{0000EE} % ???? ???????????
\definecolor{citecolor}{HTML}{000000} %???? ??? \cite
\hypersetup{linkcolor=linkcolor,urlcolor=urlcolor, citecolor=citecolor, colorlinks=true}

\ifx\pdfoutput\undefined
\usepackage{graphicx}
\else
\usepackage[pdftex]{graphicx}
\fi
\usepackage{wrapfig}

\usepackage{amssymb}
\usepackage{amsthm}
\usepackage{amsmath}

%\usepackage{euscript} % \EuScript
%\usepackage{eufrak} % \mathfrak!!
%----
\DeclareMathOperator{\Col}{Col}
\DeclareMathOperator{\Cells}{Cells}
\DeclareMathOperator{\dom}{dom}
\DeclareMathOperator{\ran}{ran}
\DeclareMathOperator{\sign}{sign}
\newtheorem{theorem}{Теорема}
\newtheorem{hypo}{Гипотеза}
\newtheorem{problem}{Задача}
\newtheorem{suite}{Следствие}
\newtheorem{lemm}{Лемма}
\newtheorem{proposition}{Предложение}
\newtheorem{examp}{Пример}
\theoremstyle{definition}
\newtheorem*{defin}{Определение}
\theoremstyle{remark}
\newtheorem{bytheway}{Замечание}

\newcommand{\samedef}{\stackrel{\mathrm{def}}{\Longleftrightarrow}}
\newcommand{\eqdef}{\stackrel{\mathrm{def}}{=}}
\newcommand{\toto}{\rightrightarrows}
\renewcommand{\Im}{\mathop{\mathrm{Im}}\nolimits}
\renewcommand{\Re}{\mathop{\mathrm{Re}}\nolimits}

\newcommand{\Z}{\mathbb{Z}}
\newcommand{\N}{\mathbb{N}}
\newcommand{\rk}{\mathrm{rk}\ }
\newcommand{\Ker}{\mathrm{Ker}\ }
\newcommand{\Lin}{\mathrm{Lin}\ }
\newcommand{\R}{\mathbb{R}}

\newcommand*{\hm}[1]{#1\nobreak\discretionary{}%
	{\hbox{$\mathsurround=0pt #1$}}{}}
\newcommand*{\hide}[1]{}

\begin{document}
	\title{Высшая школа экономики. Факультет математики.\\Итоговая государственная аттестация.}
	\maketitle
	\paragraph*{1. Числовые последовательности, пределы, предельные точки, критерий Коши сходимости последовательности.}

\begin{defin}
	Функция \(f\colon\N\to X\), областью определения которой является все множество натуральных чисел называется \textit{последовательностью} элементов множества \(X\). Значение \(f(n)\) будем обозначать как \(a_n\), а саму последовательность как \(\{a_n\}, a_n\in X\).
\end{defin}

\begin{defin}
	Число \(A\) называется \textit{пределом} последовательности \(\{a_n\}, a_n\in\R\), если для любого \(\epsilon > 0\) найдется такой номер \(N\), что для любого \(k > N\) выполнено : \[|a_k-A| < \epsilon.\] Обозначение : \(a_n\to A\) или \(\lim_{n\to\infty}a_n\hm=A\).
\end{defin}

\begin{defin}
	Число \(A\) называется \textit{пределом} последовательности \(\{a_n\}, a_n\in\R\), если для любого \(\epsilon > 0\) найдется такой номер \(N\), что для любого \(k > N\) выполнено : \[|a_k-A| < \epsilon.\]
\end{defin}
	\subsection{ Предел функции, непрерывность, теорема о промежуточном значении непрерывной функции, равномерная непрерывность непрерывной функции на отрезке.}
	\subsection{Сходимость числовых рядов. Свойства абсолютно сходящихся рядов (сходимость абсолютно сходящегося ряда, престановка членов). Признаки сходимости Д' Аламбера и Коши. Условно сходящиеся ряды. }
	\subsection{ Числовые последовательности, пределы, предельные точки, критерий Коши сходимости последовательности.}
	\subsection{парам-пам-пам}
	\input{quest06.tex}
	\input{quest07.tex}
	\input{quest08.tex}
	\input{quest09.tex}
	\input{quest10.tex}
	\input{quest11.tex}
	\input{quest12.tex}
	\input{quest13.tex}
	\input{quest14.tex}
	\input{quest15.tex}
	\input{quest16.tex}
	\input{quest17.tex}
	\input{quest18.tex}
	\input{quest19.tex}
	\input{quest20.tex}
	\input{quest21.tex}
	\input{quest22.tex}
	\input{quest23.tex}
	\input{quest24.tex}
	\input{quest24.tex}
	\input{quest25.tex}
	\input{quest26.tex}
	\input{quest27.tex}
	\input{quest28.tex}
	\input{quest29.tex}
	\input{quest30.tex}
	\input{quest31.tex}
	\input{quest32.tex}
	\input{quest33.tex}
	\input{quest34.tex}
	\input{quest35.tex}
	\input{quest36.tex}
	\input{quest37.tex}
	\input{quest38.tex}
	\input{quest39.tex}
	\input{quest40.tex}
	\input{quest41.tex}
	\input{quest42.tex}
	\input{quest43.tex}
	\input{quest44.tex}
	\input{quest45.tex}
	\input{quest46.tex}
	\input{quest47.tex}
	\input{quest48.tex}
	\input{quest49.tex}
	\input{quest50.tex}
	\input{quest51.tex}
	\input{quest52.tex}
	\input{quest53.tex}
	\input{quest54.tex}
	\input{quest55.tex}
	\input{quest56.tex}
	\input{quest56.tex}
	\input{quest57.tex}
	\input{quest58.tex}
	\input{quest59.tex}
	\input{quest60.tex}
	
\end{document}