\subsection{Числовые последовательности, пределы, предельные точки, критерий Коши сходимости последовательности.}

\begin{defin}
	Функция \(f\colon\N\to X\), областью определения которой является все множество натуральных чисел называется \textit{последовательностью} элементов множества \(X\). Значение \(f(n)\) будем обозначать как \(a_n\), а саму последовательность как \(\{a_n\}, a_n\in X\).
\end{defin}

\begin{defin}
	Число \(A\) называется \textit{пределом} последовательности \(\{a_n\}, a_n\in\R\), если для любого \(\varepsilon > 0\) найдется такой номер \(N\), что для любого \(k > N\) выполнено : \[|a_k-A| < \varepsilon.\] Обозначение : \(a_n\to A\) или \(\lim_{n\to\infty}a_n\hm=A\).
\end{defin}

\begin{defin}
	Число \(A\) называется \textit{пределом} последовательности \(\{a_n\}, a_n\in\R\), если для любого \(\varepsilon > 0\) найдется такой номер \(N\), что для любого \(k > N\) выполнено : \[|a_k-A| < \varepsilon.\]
\end{defin}

{\footnotesize \begin{proposition}[Свойства пределов последовательностей]
	Пусть \(\{a_n\}\) и \(\{b_n\}\) --- две числовые последовательности. Тогда:
	\begin{itemize}
		\item Предел последовательости единственнен, если он существует;
		
		Если кроме того, если \(a_n\to a\) и \(b_n\to b\) то:
		\item Для любого чила \(c\) выполнено \(\lim\limits_{n\to\infty}(a_n + с) = a + с \) и \(\lim\limits_{n\to\infty}(сa_n) = aс\);
		\item \(\lim\limits_{n\to\infty}(a_n + b_n) = a + b \);
		\item \(\lim\limits_{n\to\infty}(a_nb_n) = ab\);
		\item Если \(a\neq0\) и \(a_n\neq0\) для всех \(n\), то \(\frac{1}{a_n}\to\frac{1}{a}\).
	\end{itemize}
\end{proposition}

\begin{theorem}[Больцано Вейерштрасс]
	Любая ограниченная последовательность имеет по крайней мере одну предельную точку.
\end{theorem}}

\begin{defin}
	Последовательность~\(\{a_n\}\) называется \textit{фундаментальной} (последовательностью Коши), если для любого \(\varepsilon>0\) существует такой номер~\(N\), что для любых номеров~\(n, m > N\) выполнено: \(|a_n-a_m| < \varepsilon\).
\end{defin}

\begin{theorem}[критерий Коши]
	Последовательность действительных чисел \(\{x_n\}\) имеет предел тогда и только тогда, когда она является фундаментальной.
\end{theorem}
\begin{proof}
	Если \(a_n\to a\), то для \(\varepsilon>0\) выберем такой номер \(N\), что для всех номеров \(m>N\) выполнено \(|a_m-a|<\frac{\varepsilon}{2}\). Тогда для любых номеров \(k, l > N\) верно:
	\[|a_k-a_l| \leqslant |a_k - a| + |a - a_l| < \varepsilon,\]
	то есть последовательность \(\{a_n\}\) --- фундаментальна.
	
	Пусть наоборот \(\{a_n\}\) --- фундаментальная последовательность. Для фикированного \(\varepsilon\) найдем такой номер \(N\), что для каждого \(k>N\) верно \(|a_{N} - a_k|<\varepsilon\). Тогда для каждого \(n\) и \(k>N\) верно \[a_{N} - \varepsilon<a_k<a_{N} + \varepsilon.\] 
	Поэтому последовательность \(\{a_n\}\) ограничена, так как ограничен ее бесконечный "хвост".
	
	{\footnotesize Теперь можно воспользоваться Теоремой Больцано-Вейерштрасса, найти предельную точку у \(\{a_n\}\) и доказать, что найденная предельная точка и является пределом последовательности. Но мы сделаем иначе: воспользуемя принципом вложенных отрезков.}
	
	Обозначим \(l_n = \inf\limits_{k\geqslant n}a_k\) и \(u_n = \inf\limits_{k\geqslant n}a_k\). Ясно, что для любого \(n\) верно \(l_n\hm\leqslant l_{n+1}\hm\leqslant u_{n+1}\hm\leqslant u_n\). Таким образом, \(\{[l_n, u_n]\}\) --- система вложенных отрезков, а значит имеет по крайней мере одну общую точку. Кроме того, для любого \(\varepsilon\) и подходящего номера \(N\) верно:
	\[a_N - \varepsilon \leqslant l_N \leqslant u_N\leqslant a_N + \varepsilon.\]
	А значит \(\lim\limits_{n\to\infty}u_n - l_n = 0\), по лемме о двух сжимающих последовательностях. Тем самым у системы \(\{[l_n, u_n]\}\) есть ровно одна общая точка. Докажем, что она и является пределом последовательности.
	Пусть \(A = \bigcap\limits_{n=1}^{\infty}[l_n, u_n]\). Тогда для любого \(\varepsilon>0\) существует такое \(N\), что \(u_N - l_N <\varepsilon\). Тогда для любого \(k > N\) верно \(|A - a_k| \leqslant u_N - l_N < \varepsilon \), так как \(A, a_k\in[l_N, u_N]\). То есть \(a_n\to A\) при \(n\to\infty\).
\end{proof}