\paragraph*{1. Числовые последовательности, пределы, предельные точки, критерий Коши сходимости последовательности.}

\begin{defin}
	Функция \(f\colon\N\to X\), областью определения которой является все множество натуральных чисел называется \textit{последовательностью} элементов множества \(X\). Значение \(f(n)\) будем обозначать как \(a_n\), а саму последовательность как \(\{a_n\}, a_n\in X\).
\end{defin}

\begin{defin}
	Число \(A\) называется \textit{пределом} последовательности \(\{a_n\}, a_n\in\R\), если для любого \(\epsilon > 0\) найдется такой номер \(N\), что для любого \(k > N\) выполнено : \[|a_k-A| < \epsilon.\] Обозначение : \(a_n\to A\) или \(\lim_{n\to\infty}a_n\hm=A\).
\end{defin}

\begin{defin}
	Число \(A\) называется \textit{пределом} последовательности \(\{a_n\}, a_n\in\R\), если для любого \(\epsilon > 0\) найдется такой номер \(N\), что для любого \(k > N\) выполнено : \[|a_k-A| < \epsilon.\]
\end{defin}