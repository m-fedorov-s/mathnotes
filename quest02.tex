\subsection{ Предел функции, непрерывность, теорема о промежуточном значении непрерывной функции, равномерная непрерывность непрерывной функции на отрезке.}
Пусть \(f\colon E\to\R\) --- вещественнозначная функция определённая на некотором подмножестве вещественных чисел \(E\subseteq \R\). Будем считать, что \(E=\R\), если не сказано иного.
\begin{defin}
	Число \(a\) называется \textit{пределом} функции \(f\) в точке \(x_0\), если для любого \(\varepsilon>0\) можно выбрать такое \(\delta>0\), что для любой точки \(x\neq x_0\) из неравенства \(|x_0-x|<\delta\) следует неравенство \(|a - f(x)| < \varepsilon\). Обозначение: \(\lim\limits_{x\to x_0}f(x)=a\).
\end{defin}
Неформально говоря, для заранее выбранного \(\varepsilon\) все точки \(\delta\)--близкие к \(x_0\) (кроме, возможно, \(x_0\)) переходят под действием \(f\) в точки \(\varepsilon\)--близкие к \(a\).
Или формулировки в одну строчку:
\[\forall \varepsilon>0\exists\delta>0\forall x\in E : (0\neq|x-a|<\delta) \Rightarrow |f(x) - a|<\varepsilon).\]

\begin{proposition}
	Следующие утверждения эквивалентны:
	\begin{itemize}
		\item Число \(a\) является пределом функции \(f\) в точке \(x_0\).
		\item Для любой последовательности \(x_n\to x_0\), в которой не содержится элементов, равных \(x_0\), верно, что \(f(x_n)\to a\).
	\end{itemize}
\end{proposition}

\begin{proof}
	Omitted.
\end{proof}

\begin{proposition}
	Пусть \(f, g\) --- две функции, причем \(\lim\limits_{x\to x_0}f(x) \hm=a\) и \(\lim\limits_{x\to x_0}g(x)\hm=b\).
	Тогда
	\begin{itemize}
		\item \(\lim\limits_{x\to x_0}f(x)+g(x) \hm=a+b\)
		
		\item \(\lim\limits_{x\to x_0}f(x)g(x) \hm=ab\)
		
		\item \(\lim\limits_{x\to x_0}f(x)/g(x) \hm=a/b,\) если \(b\neq0.\)
	\end{itemize}
\end{proposition}
\begin{proof}
	Эти свойства моментально следуют из соответствующих свойств пределов последовательностей.
\end{proof}

\begin{defin}
	Функция \(f\) называется\textit{ непрерывной в точке }\(x_0\), если \(\lim\limits_{x\to x_0}f(x) \hm=f(x_0)\).
\end{defin}
\begin{defin}
	Функция \(f\) называется\textit{ непрерывной}, если она непрерывна в каждой точке своей области определения.
\end{defin}

\begin{theorem}[О промежуточном значении]
	Если функция \(f\) непрерывна на отрезке \([a,b]\) и принимает на его концах значения разных знаков, то существует такая точка \(x_0\in [a.b]\), что \(f(x_0) = 0\).
\end{theorem}
\begin{proof}
	Без умаления общности можно считать, что \(f(a) > 0\) и \(f(b)<0\). Постоим последовательности \(\{a_n\}, \{b_n\}\) следующим образом. Положим \(a_1\hm= a, b_1 \hm= b\).
	Для каждого натурального \(k\) рассмотрим точку \(c_k \hm= \tfrac{a_k + b_k}{2}\). Если \(f(c_k) > 0\), положим \(a_{k+1}\hm=c_k, b_{k+1}\hm=b_k\), если \(f(c_k) < 0\), то \(a_{k+1}\hm=a_k, b_{k+1}\hm=c_k\), если же \(f(c_k)=0\), то доказательство уже завершено.
	
	Таким образом мы получаем последовательность вложенных сжимающихся отрезков:
	для любого натурального \(k\) выполнено \([a_{k+1}, b_{k+1}]\subsetneq[a_k, b_k]\) и при этом \((b_n\hm-a_n)\to0\) при \(n\to\infty\).
	Тем самым у них есть ровно одна общая точка: \(x_0 = \bigcap\limits_{n=1}^{\infty}[a_n, b_n]\). Осталось проверить, что \(f(x_0) = 0\). Несложно заметить (по теореме Вейерштрасса), что \(a_n\to x_0\) и \(b_n\to x_0\) при \(n\to\infty\). Так как функция \(f\) непрерывна на отрезке \([a, b]\), то у нее существует предел в точке \(x_0\) и совпадает со значением самой функции в \(x_0\). Поэтому \(f(a_n)\to f(x_0)\) и \(f(b_n)\to f(x_0)\) при \(n\to\infty\). Но для всех \(k\) выполнено \(f(a_k) > 0\) и \(f(b_k) < 0\), откуда \(f(x_0)\geqslant0\) и \(f(x_0)\leqslant0\). А значит \(f(x_0) = 0\).
\end{proof}

\begin{defin}
	Функция \(f\) называется \textit{равномерно непрерывной} на области \(D\subseteq\R\), если для всякого \(\varepsilon>0\) можно выбрать такое \(\delta>0\), что для любых двух точек \(x, y\in D\) из неравенства \(|x-y|<\delta\) следует неравенство \(|f(x) - f(y)|<\varepsilon\).
\end{defin}

\begin{theorem}
	Если функция \(f\) непрерывна на отрезке, то она равномерно непрерывна на нем.
\end{theorem}
\begin{proof}
	Предположим, что функция \(f\) не равномерно непрерывна на отрезке \([a,b]\). Тогда существует такой \(\varepsilon_0>0\), что для любого  натурального \(k\) существуют такие \(x_k, y_k\), что \(|x_k-y_k| < \tfrac{1}{k}\) и \(|f(x_k) - f(y_k)|>\varepsilon_0\). Тогда из последовательности \(\{x_k\}\) можно выбрать сходящуюся подпоследовательность \(x_{k_j}\to x_0\). Так как \(|x_{k_j}-y_{k_j}|\to0\) то и \(y_{k_j}\to x_0\). Тогда, в силу непрерывности функции \(f\) последовательности ее значений в точках последовательностей \(\{x_{k_j}\}\) и \(\{y_{k_j}\}\) сходятся: \(f(x_{k_j})\to f(x_0)\) и \(f(y_{k_j})\to f(x_0)\). Но это противоречит неравенству \(|f(x_k) - f(y_k)|>\varepsilon_0\).
\end{proof}